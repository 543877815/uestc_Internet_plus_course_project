%%%%%%%%%%%%%%%%%%%%%%%%%%%%%%%%%%%%%%%%%%%%%%%%%%%%%%%%%%%%%%%%%%%%%
%
% CSCI 1430 Writeup Template
%
% This is a LaTeX document. LaTeX is a markup language for producing
% documents. Your task is to fill out this
% document, then to compile this into a PDF document.
%
% TO COMPILE:
% > pdflatex thisfile.tex
%
% For references to appear correctly instead of as '??', you must run
% pdflatex twice.
%
% If you do not have LaTeX and need a LaTeX distribution:
% - Departmental machines have one installed.
% - Personal laptops (all common OS): www.latex-project.org/get/
%
% If you need help with LaTeX, please come to office hours.
% Or, there is plenty of help online:
% https://en.wikibooks.org/wiki/LaTeX
%
% Good luck!
% James and the 1430 staff
%
%%%%%%%%%%%%%%%%%%%%%%%%%%%%%%%%%%%%%%%%%%%%%%%%%%%%%%%%%%%%%%%%%%%%%
%
% How to include two graphics on the same line:
%
% \includegraphics[\width=0.49\linewidth]{yourgraphic1.png}
% \includegraphics[\width=0.49\linewidth]{yourgraphic2.png}
%
% How to include equations:
%
% \begin{equation}
% y = mx+c
% \end{equation}
%
%%%%%%%%%%%%%%%%%%%%%%%%%%%%%%%%%%%%%%%%%%%%%%%%%%%%%%%%%%%%%%%%%%%%%%%%%%%%%%%%%%%%%%%%%%%%%%%%

\documentclass[11pt]{article}

\usepackage[english]{babel}
\usepackage[utf8]{inputenc}
\usepackage[colorlinks = true,
            linkcolor = blue,
            urlcolor  = blue]{hyperref}
\usepackage[a4paper,margin=1.5in]{geometry}
\usepackage{stackengine,graphicx}
\usepackage{fancyhdr}
\setlength{\headheight}{15pt}
\usepackage{microtype}
\usepackage{times}
\usepackage{booktabs}

% python code format: https://github.com/olivierverdier/python-latex-highlighting
\usepackage{pythonhighlight}

\frenchspacing
\setlength{\parindent}{0cm} % Default is 15pt.
\setlength{\parskip}{0.3cm plus1mm minus1mm}

\pagestyle{fancy}
\fancyhf{}
\lhead{Project 3 Writeup}
\rhead{CSCI 1430}
\rfoot{\thepage}

\date{}

\title{\vspace{-1cm}Project 3 Writeup}


\begin{document}
\maketitle
\vspace{-2cm}
\thispagestyle{fancy}

\section*{Instructions}
\begin{itemize}
  \item Describe any interesting decisions you made to write your algorithm.
  \item Show and discuss the results of your algorithm.
  \item Feel free to include code snippets, images, and equations.
  \item Use as many pages as you need, but err on the short side.
  \item \textbf{Please make this document anonymous.}
\end{itemize}

\section*{In the beginning...}

Lorem ipsum dolor sit amet, consectetur adipisicing elit, sed do eiusmod tempor incididunt ut labore et dolore magna aliqua. Ut enim ad minim veniam, quis nostrud exercitation ullamco laboris nisi ut aliquip ex ea commodo consequat. Duis aute irure dolor in reprehenderit in voluptate velit esse cillum dolore eu fugiat nulla pariatur. Excepteur sint occaecat cupidatat non proident, sunt in culpa qui officia deserunt mollit anim id est laborum. See Equation~\ref{eq:one}.

\begin{equation}
a = b + c
\label{eq:one}
\end{equation}

\section*{Interesting Implementation Detail}

Lorem ipsum dolor sit amet, consectetur adipisicing elit, sed do eiusmod tempor incididunt ut labore et dolore magna aliqua. Ut enim ad minim veniam, quis nostrud exercitation ullamco laboris nisi ut aliquip ex ea commodo consequat. Duis aute irure dolor in reprehenderit in voluptate velit esse cillum dolore eu fugiat nulla pariatur. Excepteur sint occaecat cupidatat non proident, sunt in culpa qui officia deserunt mollit anim id est laborum.

My code snippet highlights an interesting point.
\begin{python}
one = 1;
two = one + one;
if two == 2
    disp( 'This computer is not broken.' );
end
\end{python}

\section*{A Result}

\begin{enumerate}
    \item Result 1 was a total failure, because...
    \item Result 2 (Figure~\ref{fig:result1}, left) was surprising, because...
    \item Result 3 (Figure~\ref{fig:result1}, right) blew my socks off, because...
\end{enumerate}

\begin{figure}[h]
    \centering
    \includegraphics[width=5cm]{placeholder.jpg}
    \includegraphics[width=5cm]{placeholder.jpg}
    \caption{\emph{Left:} My result was spectacular. \emph{Right:} Curious.}
    \label{fig:result1}
\end{figure}

My results are summarized in Table~\ref{tab:table1}.

\begin{table}[h]
    \centering
    \begin{tabular}{lr}
        \toprule
        Condition & Time (seconds) \\
        \midrule
        Test 1 & 1 \\
        Test 2 & 1000 \\
        \bottomrule
    \end{tabular}
    \caption{Stunning revelation about the efficiency of my code.}
    \label{tab:table1}
\end{table}

\end{document}
